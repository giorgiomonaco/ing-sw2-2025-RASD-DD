\section{Scope}
\label{sec:scope}%

This document outlines the domain and objectives of the system, focusing on facilitating interactions and processes among users, including students, companies, and universities. It reviews the high-level architectural design, emphasizing the adoption of established architectural styles such as multi-tier architecture and modular patterns. The chosen design ensures scalability, security, and maintainability while addressing key functional and non-functional requirements of the system. Additionally, this document provides a framework for understanding how the components and subsystems integrate to deliver a cohesive and efficient solution.

\section{Definitions, Acronyms, Abbreviations}
\label{sec:definition_acronyms_abbreviations}%

\subsection{Definitions}
\label{subsec:definitions}%

\begin{itemize}
\item 
  \textbf{Software Architecture}: The set of structures needed to reason about the system. These structures comprise software elements, relations among them and properties of both.
\item
  \textbf{Recommendation}: The process of identifying and suggesting internships to students and suitable candidates to companies using keyword searches, statistical analyses, or other matching algorithms.
\item
  \textbf{Selection Process}: A structured process following a contact in which companies interview and assess students to determine fit.
\item
  \textbf{Matching}: The automated or semi-automated process of pairing students with internships based on their profiles and internship requirements.
\item
  \textbf{Feedback}: Information provided by students and companies on the quality of the matchmaking process or the internship experience, used to improve recommendations.
\item
  \textbf{Internship Monitoring}: Ongoing observation on internship status.
\item
  \textbf{Users:} referred to logged-in guests: Universities, Companies and Students.
\item
  \textbf{Guest:} non logged visitors.
\item
  \textbf{S\&C Calendar:} a built-in calendar where all events (start and end date of internship, interviews) are visible and scheduled.
\end{itemize}

\subsection{Acronyms}
\label{subsec:acronyms}%

\begin{itemize}
\item
  \textbf{S\&C}: Students \& Companies.
\item
  \textbf{UI}: User Interface.
\item
  \textbf{API}: Application Programming Interface.
\end{itemize}

\subsection{Abbreviations}
\label{subsec:abbreviations}%

\begin{itemize}
\item
  \textbf{[R*]:} Functional Requirement.
\item
  \textbf{[UC*]:} Use Case.
\item
  \textbf{[S]} Student.
\item
  \textbf{[C]}: Company.
\item
  \textbf{[U]}: University.
\end{itemize}


% First version, no revision
%\section{Revision history}
%\label{sec:revision_history}%


\section{Reference Documents}
\label{sec:reference_documents}%

The document is based on the following materials:

\begin{itemize}
\item
  The specification of the RASD and DD assignment of the Software Engineering 2 course a.a 2024/2025.
\item
  Slides of the course on WeBeep.
\end{itemize}

\section{Document Structure}
\label{sec:document_structure}%

\paragraph{Introduction:} Provides an overview of the system's goals, scope, and purpose, setting the context for the following sections.
  
\paragraph{Architectural Design:} Describes the architectural styles, patterns, and components of the system, including the detailed organization of layers and the interactions among the components.

\paragraph{User Interface Design:} Presents the main user interface designs for meaningful use cases, showcasing how users interact with the system.

\paragraph{Requirements Traceability:} Maps the functional and non-functional requirements from the RASD document to the specific components and subsystems described in the architectural design.

\paragraph{Implementation, Integration and Test Plan:} Details the system implementation steps, component integration, and testing strategy, including unit, integration, and system tests to ensure requirements are met.

\paragraph{Effort Spent:} Overview of the team's time allocation for each document section.

\paragraph{References:} Bibliography listing all resources, documentation and software used in preparing this document.