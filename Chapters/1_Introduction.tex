\section{Purpose}
\label{sec:purpose}%

As the demand for skilled graduates grows, bridging the gap between academic learning and practical industry experience has become essential. However, students often face challenges in finding internships that align with their skills and career goals, while companies struggle to identify candidates who truly match their specific needs. Traditional job boards and application processes don’t cater to the nuanced requirements of internship placements, leading to mismatches and missed opportunities for both students and employers.
Students and companies (S\&C) exists to address this gap by providing a tailored matchmaking platform. By aligning students’ skills, experiences, and aspirations with detailed internship roles, S\&C ensures better fit and smoother transitions into the workforce. The platform supports proactive searches, personalized recommendations, and structured selection processes, ultimately helping students gain hands-on experience while allowing companies to find the right talent for their projects. S\&C not only simplifies the internship process but also provides tools for feedback, monitoring, and communication, creating a valuable, structured internship experience that benefits students, companies, and universities alike.


\subsection{Goals}
\label{subsec:goals}%

{[}G1{]}: University students would like to find internships that better
align with their interests and field of study.

{[}G2{]}: Companies would like to offer internships to students who best
match the related figure.

{[}G3{]}: Improve communication and coordination between students and
companies in the selection process.

{[}G4{]}: Evolve the recommendation system through feedback and
suggestions (provided by students and companies) to improve future
matches.

{[}G5{]}: Universities would like to monitor internships and handle any
emerging issues promptly, ensuring a smooth and beneficial experience
for all parties involved.

\section{Scope}
\label{sec:scope}%

In this section, we are going to identify the S\&C platform domain. There
are three main users that interact with the system:~

\begin{enumerate}
\def\labelenumi{\arabic{enumi}.}
\item
  \textbf{Students} -- University students seeking internships as part of their academic or career development. They can use S\&C to
  proactively browse available internships, receive customized recommendations based on their profiles, and interact with companies to begin the application and selection process.
\item
  \textbf{Companies} -- Businesses and organizations that offer internships. These companies use S\&C to advertise internship roles with detailed information on project tasks, required skills,
  technologies, and benefits. S\&C enables companies to receive suitable student profiles automatically, simplifying their candidate search and supporting the structured interview and selection process.
\item
  \textbf{Universities} -- Academic institutions that need to monitor
  and oversee students' internship experiences. Universities use S\&C to
  track the status and progress of internships, address student
  concerns, and intervene in issues that may impact the quality of the
  internship.
\end{enumerate}

According to the World and Machine paradigm we can identify the Machine
as the System to be developed and the environment in which S\&C will be
used as the World. The separation between these two concepts allows us
to classify the entire phenomena in three different types.

\subsection{World Phenomena}
\label{subsec:world_phenomena}%

Events that take place in the real world and that the machine cannot
observe:

{[}WP1{]} The company creates and defines internship opportunities
independently, describing tasks, required skills, and terms.

{[}WP2{]} The student creates CVs detailing his skills and attitudes.

{[}WP3{]} The company conducts an interview with the student.

{[}WP4{]} The student wants to accept / reject the proposal.

{[}WP5{]} The student or the company encounters problems during the
internship and wants to report it.~

{[}WP6{]} The university handles the complaint submitted by his student
or a related company.

\subsection{Machine Phenomena}
\label{subsec:machine_phenomena}%

Events that take place inside the System and cannot be observed by the
real world.

{[}MP1{]} The internal processes analyze student profiles and internship
postings to generate recommendations.~

{[}MP2{]} The recommendation system automatically improves using
feedback and suggestions.

\subsection{Shared phenomena}
\label{subsec:shared_phenomena}%

\begin{itemize}
\item
  \textbf{World controlled}
\end{itemize}

Controlled by the world and observed by the machine.

{[}SP1{]} The S /C signs up / logs in to the system.

{[}SP2{]} The C uploads internship details, terms, and benefits, which
the system then collects.

{[}SP3{]} The S improves/edits his CV based on the suggestions.

{[}SP4{]} The C improves/edits his internship insertion based on the
suggestions.

{[}SP5{]} The S proactively looks for an internship on the platform, by
querying a company\textquotesingle s name or a specific position.

{[}SP6{]} The S reviews a C's profile

{[}SP8{]} The S reviews his calendar.

{[}SP6{]} The S surf through the recommended internships~

{[}SP7{]} The S reviews an internship insertion

{[}SP7{]} The S or the C submits feedback and suggestions about the
matchmaking process.

{[}SP9{]} The S applies for an internship.

{[}SP10{]}The C reviews a candidate's profile

{[}SP11{]}The C sends a contact request in order to offer an internship
to a recommended student.

{[}SP12{]} The S accepts/refuses a contact request by a company.

{[}SP13{]} The C and the S schedule an interview through the chat

{[}SP6{]} The S and the C finalize the selection process after the
interview.~

{[}SP{]} The C offers an internship contract to the student after the
interview.

{[}SP{]}The S accepts/refuses the proposed contract.

{[}SP8{]}The S or the C submits problems, complaints and information
about the ongoing internship.~

{[}SP9{]} The U monitors the ongoing internship.

{[}SP10{]}The U interrupts the internship.

\begin{itemize}
\item
  \textbf{Machine controlled}
\end{itemize}

Controlled by the machine and observed by the World.

{[}SP17{]} The system analyzes the CV and suggests to the S how to
improve it.

{[}SP18{]} The system analyzes the Insertion and suggests to the C how
to improve it.

{[}SP11{]} The system recommends the best internships to the S based on
his Cv, his attitudes and previously submitted feedback.

{[}SP12{]} The system suggests the best candidates to the company based
on matching algorithms and keywords.

{[}SP13{]} The system notifies the S when new recommended internships
become available by sending an email and a notification.

{[}SP14{]} The system notifies the C of potential candidates when new
CVs become available by sending an email and a notification.

{[}SP20{]} The system asks the S and the company to provide feedback or
suggestions about the matchmaking process.

{[}SP15{]} The system establishes contact between the S and the C when
both parties have accepted each other's request.

{[}SP21{]} The system adds to the S\&C calendar the start and the end
date of the internship after the contract agreement.

{[}SP22{]} The system adds to the S\&C registers in the calendar the
submission of information, complaint or problem.

\section{Definitions, Acronyms, Abbreviations}
\label{sec:definition_acronyms_abbreviations}%

\subsection{Definitions}
\label{subsec:definitions}%

\begin{itemize}
\item
  \textbf{Recommendation}: The process of identifying and suggesting
  internships to students and suitable candidates to companies using
  keyword searches, statistical analyses, or other matching algorithms.
\item
  \textbf{Selection Process}: A structured process following a
  recommendation in which companies interview and assess students to
  determine fit.
\item
  \textbf{Matching}: The automated or semi-automated process of pairing
  students with internships based on their profiles and internship
  requirements.
\item
  \textbf{Feedback}: Information provided by students and companies on
  the quality of the matchmaking process or the internship experience,
  used to improve recommendations.
\item
  \textbf{Internship Monitoring}: Ongoing observation on internship
  status.
\item
  \textbf{Users:} referred to logged-in guests: Universities, Companies
  and Students
\item
  \textbf{Guest:} non logged visitors.
\item
  \textbf{S\&C Calendar:} a built-in calendar where all events
  (submissions, start and end date of internship, interviews) are
  visible and scheduled.~
\end{itemize}

\subsection{Acronyms}
\label{subsec:acronyms}%

\begin{itemize}
\item
  \textbf{S\&C}: Students \& Companies
\item
  \textbf{UI}: User Interface
\item
  \textbf{API}: Application Programming Interface
\end{itemize}

\subsection{Abbreviations}
\label{subsec:abbreviations}%

\begin{itemize}
\item
  \textbf{{[}G*{]}}: Goal
\item
  \textbf{{[}D*{]}}: Domain Assumption
\item
  \textbf{{[}R*{]}}: Functional Requirement
\item
  \textbf{{[}WP*{]}}: World Phenomena
\item
  \textbf{{[}MP*{]}}: Machine Phenomena
\item
  \textbf{{[}SP*{]}}: Shared Phenomena
\item
  \textbf{{[}UC*{]}}: Use Case
\item
  \textbf{{[}S{]}}: Student
\item
  \textbf{{[}C{]}}: Company
\item
  \textbf{{[}U{]}}: University
\end{itemize}

\section{Revision history}
\label{sec:revision_history}%

da aggiungere alla fine

\section{Reference Documents}
\label{sec:reference_documents}%

The document is based on the following materials:

\begin{itemize}
\item
  The specification of the RASD and DD assignment of the Software
  Engineering II course a.a 2024/2025~
\item
  Slides of the course on WeBeep
\end{itemize}

\section{Document Structure}
\label{sec:document_structure}%

\paragraph{Introduction}: It aims to give an overview of the project. In
  particular it's focused on the reasons and goals that are going to be
  achieved with its development.
\paragraph{Overall Description}: This section provides a high-level
  overview of how the S\&C platform operates, describing the roles and
  interactions of the main users: students, companies, and universities.
  It categorizes the phenomena in the system using the World and Machine
  paradigm and outlines assumptions and dependencies within the
  platform's domain.
\paragraph{Specific Requirements}: Detailed functional and non-functional
  requirements for achieving the goals. Moreover, it contains more
  information useful for developers (i.e constraints about HW and SW)
\paragraph{Formal Analysis}: Formal modeling of the key phenomena using
  Alloy.
\paragraph{Effort Spent}: Overview of the team's time allocation for each
  document section.
\paragraph{References}: Bibliography listing all resources,
  documentation, and software used in preparing this document.